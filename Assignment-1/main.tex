\documentclass{article}
\usepackage[utf8]{inputenc}
\usepackage[english]{babel}

\usepackage{multicol}

\title{AI1103 Assignment 1}
\author{Megh Shah - CS20BTECH11032}
\date{}

\begin{document}

\maketitle
\begin{multicols*}{2}
\section*{}

\underline {Problem 2.18}: \newline \newline A man is known to speak truth 3 out of 4 times. He throws a die and reports that it is a six. Find the probability that it is actually a six.
\newline \newline \underline {Solution 2.18}: \newline \newline Let $\Pr(x=i)$ denote the probability that the number $i$ is obtained on the die.
\newline Let $\Pr(y=i)$ denote the probability that the $i$ is reported as the number on the die.
\newline Let $\Pr(z=0)$ denote the probability the man is lying and $\Pr(z=1)$ denotes the probability that the man is telling the truth.
\newline \newline The notation used here is:
\newline $\Pr(A=i\cdot B=j)\equiv \Pr(A=i \land B=j)$
\newline \newline Now, we have to find out: $$\Pr(x=6|y=6)$$
Recalling Bayes' Theorem: 
$$ \Pr(A|B)=\frac{\Pr(AB)}{\Pr(B)} ... [1] $$
Now, $\Pr(x=6 \cdot y=6)$ is only possible when the man is telling the truth (z=1) and the die rolls a 6 (x=6)
\newline $\Pr(x=6 \cdot y=6)=\Pr(x=6 \cdot z=1)$
\newline Both of these are independent events, hence by definition:
$$\Pr(x=6 \cdot z=1)=\Pr(x=6)\Pr(z=1) $$
$$\Pr(x=6 \cdot z=1)=(1/6)*(3/4)=1/8 $$
Hence, we have: 
\newline $\Pr(x=6 \cdot y=6)=1/8 $
\newline \newline Now for $\Pr(y=6)$:
\newline We know by symmetry that $$\Pr(y=i) = \Pr(y=j) ... [2]$$ $\forall$ $i,j \in \{ 1,2,3,4,5,6 \}$ 
\newline \newline Also, since these are all disjoint cases whose union covers all cases, we also have:
\newline $$ \Pr(y=1)+\Pr(y=2)+...\Pr(y=6) = 1 $$
\newline From [2], we have 
$$ \Pr(y=6)+\Pr(y=6)...\Pr(y=6) = 1 $$
$$ 6\Pr(y=6) = 1 $$
$$ \Pr(y=6) = 1/6 $$
Putting the obtained results back in [1],
$$\Pr(x=6|y=6) = \frac{\Pr(x=6 \cdot y=6)}{\Pr(y=6)} $$
$$\Pr(x=6|y=6) = \frac{1/8}{1/6} = 3/4$$
Hence, the required probability is 0.75
\end{multicols*}
\end{document}