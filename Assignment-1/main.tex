\documentclass[journal,12pt,twocolumn]{IEEEtran}

\usepackage{setspace}
\usepackage{gensymb}
\singlespacing
\usepackage[cmex10]{amsmath}

\usepackage{amsthm}

\usepackage{mathrsfs}
\usepackage{txfonts}
\usepackage{stfloats}
\usepackage{bm}
\usepackage{cite}
\usepackage{cases}
\usepackage{subfig}

\usepackage{longtable}
\usepackage{multirow}

\usepackage{enumitem}
\usepackage{mathtools}
\usepackage{steinmetz}
\usepackage{tikz}
\usepackage{circuitikz}
\usepackage{verbatim}
\usepackage{tfrupee}
\usepackage[breaklinks=true]{hyperref}
\usepackage{graphicx}
\usepackage{tkz-euclide}

\usetikzlibrary{calc,math}
\usepackage{listings}
    \usepackage{color}                                            %%
    \usepackage{array}                                            %%
    \usepackage{longtable}                                        %%
    \usepackage{calc}                                             %%
    \usepackage{multirow}                                         %%
    \usepackage{hhline}                                           %%
    \usepackage{ifthen}                                           %%
    \usepackage{lscape}     
\usepackage{multicol}
\usepackage{chngcntr}

\DeclareMathOperator*{\Res}{Res}

\renewcommand\thesection{\arabic{section}}
\renewcommand\thesubsection{\thesection.\arabic{subsection}}
\renewcommand\thesubsubsection{\thesubsection.\arabic{subsubsection}}

\renewcommand\thesectiondis{\arabic{section}}
\renewcommand\thesubsectiondis{\thesectiondis.\arabic{subsection}}
\renewcommand\thesubsubsectiondis{\thesubsectiondis.\arabic{subsubsection}}


\hyphenation{op-tical net-works semi-conduc-tor}
\def\inputGnumericTable{}                                 %%

\lstset{
%language=C,
frame=single, 
breaklines=true,
columns=fullflexible
}
\begin{document}

\newcommand{\BEQA}{\begin{eqnarray}}
\newcommand{\EEQA}{\end{eqnarray}}
\newcommand{\define}{\stackrel{\triangle}{=}}
\bibliographystyle{IEEEtran}
\raggedbottom
\setlength{\parindent}{0pt}
\providecommand{\mbf}{\mathbf}
\providecommand{\pr}[1]{\ensuremath{\Pr\left(#1\right)}}
\providecommand{\qfunc}[1]{\ensuremath{Q\left(#1\right)}}
\providecommand{\sbrak}[1]{\ensuremath{{}\left[#1\right]}}
\providecommand{\lsbrak}[1]{\ensuremath{{}\left[#1\right.}}
\providecommand{\rsbrak}[1]{\ensuremath{{}\left.#1\right]}}
\providecommand{\brak}[1]{\ensuremath{\left(#1\right)}}
\providecommand{\lbrak}[1]{\ensuremath{\left(#1\right.}}
\providecommand{\rbrak}[1]{\ensuremath{\left.#1\right)}}
\providecommand{\cbrak}[1]{\ensuremath{\left\{#1\right\}}}
\providecommand{\lcbrak}[1]{\ensuremath{\left\{#1\right.}}
\providecommand{\rcbrak}[1]{\ensuremath{\left.#1\right\}}}
\theoremstyle{remark}
\newtheorem{rem}{Remark}
\newcommand{\sgn}{\mathop{\mathrm{sgn}}}
\providecommand{\abs}[1]{\vert#1\vert}
\providecommand{\res}[1]{\Res\displaylimits_{#1}} 
\providecommand{\norm}[1]{\lVert#1\rVert}
%\providecommand{\norm}[1]{\lVert#1\rVert}
\providecommand{\mtx}[1]{\mathbf{#1}}
\providecommand{\mean}[1]{E[ #1 ]}
\providecommand{\fourier}{\overset{\mathcal{F}}{ \rightleftharpoons}}
%\providecommand{\hilbert}{\overset{\mathcal{H}}{ \rightleftharpoons}}
\providecommand{\system}{\overset{\mathcal{H}}{ \longleftrightarrow}}
	%\newcommand{\solution}[2]{\textbf{Solution:}{#1}}
\newcommand{\solution}{\noindent \textbf{Solution: }}
\newcommand{\cosec}{\,\text{cosec}\,}
\providecommand{\dec}[2]{\ensuremath{\overset{#1}{\underset{#2}{\gtrless}}}}
\newcommand{\myvec}[1]{\ensuremath{\begin{pmatrix}#1\end{pmatrix}}}
\newcommand{\mydet}[1]{\ensuremath{\begin{vmatrix}#1\end{vmatrix}}}
\numberwithin{equation}{subsection}
\makeatletter
\@addtoreset{figure}{problem}
\makeatother
\let\StandardTheFigure\thefigure
\let\vec\mathbf
\renewcommand{\thefigure}{\theproblem}
\def\putbox#1#2#3{\makebox[0in][l]{\makebox[#1][l]{}\raisebox{\baselineskip}[0in][0in]{\raisebox{#2}[0in][0in]{#3}}}}
     \def\rightbox#1{\makebox[0in][r]{#1}}
     \def\centbox#1{\makebox[0in]{#1}}
     \def\topbox#1{\raisebox{-\baselineskip}[0in][0in]{#1}}
     \def\midbox#1{\raisebox{-0.5\baselineskip}[0in][0in]{#1}}
\vspace{3cm}
\title{AI1103 Assignment 1}
\author{Megh Shah - CS20BTECH11032}
\maketitle
\newpage
\bigskip
\renewcommand{\thefigure}{\theenumi}
\renewcommand{\thetable}{\theenumi}
Download all latex-tikz codes from: 
\begin{lstlisting}
https://github.com/MShah134/AI1103/blob/main/Assignment-1/main.tex
\end{lstlisting}
\section*{Problem}
A man is known to speak truth 3 out of 4 times. He throws a die and reports that it is a six. Find the probability that it is actually a six.
\section*{Solution}
\begin{enumerate}
    \item Let $\Pr(X=i)$ be the probability that number $i$ is obtained on the die.
    \item Let $\Pr(Y=i)$ be the probability that number $i$ is reported on the die.
    \item Let $\Pr(Z=0)$ be the probability the man is lying.
    \item Let $\Pr(Z=1)$ be the probability that the man is telling the truth.
\end{enumerate}
We have to find: $\Pr(X=6|Y=6)$
\begin{align}
\Pr(A|B)=\frac{\Pr(AB)}{\Pr(B)} 
\label{Bayes}
\end{align}
Now, $ X=6, Y=6 \implies X=6, Z=1$ 
\newline (man is telling the truth and the die rolls a six)
\newline \newline $\Pr(X=6, Y=6)=\Pr(X=6, Z=1)$
\newline Since these are independent events:
\begin{align}
\Pr(X=6, Z=1)=\Pr(X=6)\Pr(Z=1) \\
\Pr(X=6, Z=1)=(1/6) \times (3/4)=1/8
\end{align}
$\implies \Pr(X=6, Y=6)=1/8 $
\newline \newline By symmetry we get, 
\begin{align}
   \Pr(Y=i) = \Pr(Y=j)
   \label{equal}
\end{align}
$\forall$ $i,j \in \{ 1,2,3,4,5,6 \}$ 
\newline \newline Also, since these are all disjoint cases whose union is 1, we get:
\begin{align}
    \sum_{i=1}^{6} \Pr(Y=i) = 1 
    \label{sum}
\end{align}
From eq \ref{equal} and \ref{sum}, we have 
$$ \Pr(Y=6) = 1/6 $$
Putting the obtained results back in eq \ref{Bayes},
$$\Pr(X=6|Y=6) = \frac{\Pr(X=6, Y=6)}{\Pr(Y=6)} $$
$$\Pr(X=6|Y=6) = \frac{1/8}{1/6} = 3/4$$
Hence, the required probability is 3/4.
\newline \newline Consider table \ref{table1}:
\begin{center}
\begin{table}[h]
    \centering
    \begin{tabular}{ |m{1.75cm}| m{1.75cm}| m{2cm} | } 
\hline
Case(s) & Notes & Probability\\ 
\hline
$\Pr(X=i)$ & i=1,2,...6 & 1/6 \\ 
\hline
$\Pr(Y=i)$ & i=1,2,...6 & 1/6 \\ 
\hline
$\Pr(Z=0)$ & - & 1/4 \\ 
\hline 
$\Pr(Z=1)$ & - & 3/4 \\ 
\hline
\end{tabular}
    \caption{Probabilities of random variables}
    \label{table1}
\end{table}
\end{center}
\end{document}
