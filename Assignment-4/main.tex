\documentclass[journal,12pt,twocolumn]{IEEEtran}

\usepackage{setspace}
\usepackage{gensymb}
\singlespacing
\usepackage[cmex10]{amsmath}

\usepackage{amsthm}

\usepackage{mathrsfs}
\usepackage{txfonts}
\usepackage{stfloats}
\usepackage{bm}
\usepackage{cite}
\usepackage{cases}
\usepackage{subfig}

\usepackage{longtable}
\usepackage{multirow}
\usepackage{caption}

\usepackage{enumitem}
\usepackage{mathtools}
\usepackage{steinmetz}
\usepackage{tikz}
\usepackage{circuitikz}
\usepackage{verbatim}
\usepackage{tfrupee}
\usepackage[breaklinks=true]{hyperref}
\usepackage{graphicx}
\usepackage{tkz-euclide}

\usetikzlibrary{calc,math}
\usepackage{listings}
    \usepackage{color}                                            %%
    \usepackage{array}                                            %%
    \usepackage{longtable}                                        %%
    \usepackage{calc}                                             %%
    \usepackage{multirow}                                         %%
    \usepackage{hhline}                                           %%
    \usepackage{ifthen}                                           %%
    \usepackage{lscape}     
\usepackage{multicol}
\usepackage{chngcntr}

\DeclareMathOperator*{\Res}{Res}

\renewcommand\thesection{\arabic{section}}
\renewcommand\thesubsection{\thesection.\arabic{subsection}}
\renewcommand\thesubsubsection{\thesubsection.\arabic{subsubsection}}

\renewcommand\thesectiondis{\arabic{section}}
\renewcommand\thesubsectiondis{\thesectiondis.\arabic{subsection}}
\renewcommand\thesubsubsectiondis{\thesubsectiondis.\arabic{subsubsection}}

\newcommand{\E}{\mathrm{E}}
\newcommand{\Var}{\mathrm{Var}}

\hyphenation{op-tical net-works semi-conduc-tor}
\def\inputGnumericTable{}                                 %%

\lstset{
%language=C,
frame=single, 
breaklines=true,
columns=fullflexible
}
\begin{document}

\newcommand{\BEQA}{\begin{eqnarray}}
\newcommand{\EEQA}{\end{eqnarray}}
\newcommand{\define}{\stackrel{\triangle}{=}}
\bibliographystyle{IEEEtran}
\raggedbottom
\setlength{\parindent}{0pt}
\providecommand{\mbf}{\mathbf}
\providecommand{\pr}[1]{\ensuremath{\Pr\left(#1\right)}}
\providecommand{\qfunc}[1]{\ensuremath{Q\left(#1\right)}}
\providecommand{\sbrak}[1]{\ensuremath{{}\left[#1\right]}}
\providecommand{\lsbrak}[1]{\ensuremath{{}\left[#1\right.}}
\providecommand{\rsbrak}[1]{\ensuremath{{}\left.#1\right]}}
\providecommand{\brak}[1]{\ensuremath{\left(#1\right)}}
\providecommand{\lbrak}[1]{\ensuremath{\left(#1\right.}}
\providecommand{\rbrak}[1]{\ensuremath{\left.#1\right)}}
\providecommand{\cbrak}[1]{\ensuremath{\left\{#1\right\}}}
\providecommand{\lcbrak}[1]{\ensuremath{\left\{#1\right.}}
\providecommand{\rcbrak}[1]{\ensuremath{\left.#1\right\}}}
\theoremstyle{remark}
\newtheorem{rem}{Remark}
\newcommand{\sgn}{\mathop{\mathrm{sgn}}}
\providecommand{\abs}[1]{\vert#1\vert}
\providecommand{\res}[1]{\Res\displaylimits_{#1}} 
\providecommand{\norm}[1]{\lVert#1\rVert}
%\providecommand{\norm}[1]{\lVert#1\rVert}
\providecommand{\mtx}[1]{\mathbf{#1}}
\providecommand{\mean}[1]{E[ #1 ]}
\providecommand{\fourier}{\overset{\mathcal{F}}{ \rightleftharpoons}}
%\providecommand{\hilbert}{\overset{\mathcal{H}}{ \rightleftharpoons}}
\providecommand{\system}{\overset{\mathcal{H}}{ \longleftrightarrow}}
	%\newcommand{\solution}[2]{\textbf{Solution:}{#1}}
\newcommand{\solution}{\noindent \textbf{Solution: }}
\newcommand{\cosec}{\,\text{cosec}\,}
\providecommand{\dec}[2]{\ensuremath{\overset{#1}{\underset{#2}{\gtrless}}}}
\newcommand{\myvec}[1]{\ensuremath{\begin{pmatrix}#1\end{pmatrix}}}
\newcommand{\mydet}[1]{\ensuremath{\begin{vmatrix}#1\end{vmatrix}}}
\numberwithin{equation}{subsection}
\makeatletter
\@addtoreset{figure}{problem}
\makeatother
\let\StandardTheFigure\thefigure
\let\vec\mathbf
\renewcommand{\thefigure}{\theproblem}
\def\putbox#1#2#3{\makebox[0in][l]{\makebox[#1][l]{}\raisebox{\baselineskip}[0in][0in]{\raisebox{#2}[0in][0in]{#3}}}}
     \def\rightbox#1{\makebox[0in][r]{#1}}
     \def\centbox#1{\makebox[0in]{#1}}
     \def\topbox#1{\raisebox{-\baselineskip}[0in][0in]{#1}}
     \def\midbox#1{\raisebox{-0.5\baselineskip}[0in][0in]{#1}}
\vspace{3cm}
\title{AI1103 Assignment 3}
\author{Megh Shah - CS20BTECH11032}
\maketitle
\newpage
\bigskip
\renewcommand{\thefigure}{\theenumi}
\renewcommand{\thetable}{\theenumi}
Download all latex-tikz codes from 
%
\begin{lstlisting}
https://github.com/MShah134/AI1103/blob/main/Assignment-4/main.tex
\end{lstlisting}
\section*{Question}
$X_1$ and $X_2$ are independent Poisson variables such that $\pr{X_1=2} = \pr{X_1=1}$ and $\pr{X_2=2} = \pr{X_2=3}$. What is the variance of $(X_1 - 2X_2)$ ?
\newline (a) 14
\newline (b) 4
\newline (c) 3
\newline (d) 2
\section*{Solution}
For a Poisson variable X,
\begin{align}
\pr{X=k} = \frac{\lambda^{k}e^{-\lambda}}{k!}
\end{align}
Since $\pr{X_1=2} = \pr{X_1=1}$,
\begin{align}
\frac{{\lambda_1}^{2}e^{-{\lambda_1}}}{2!} &= \frac{{\lambda_1}^{1}e^{-{\lambda_1}}}{1!} \\
\lambda_1 &= 2!/1! = 2
\end{align}
Similarly, as $\pr{X_2=2} = \pr{X_2=3}$,
\begin{align}
\frac{{\lambda_2}^{2}e^{-{\lambda_2}}}{2!} &= \frac{{\lambda_2}^{3}e^{-{\lambda_2}}}{3!} \\
\lambda_2 &= 3!/2! = 3
\end{align}
Also we know for a Poisson variable X, the following holds true:
\begin{align}
\E[X] &= \lambda \\
\Var[X] &= \lambda \label{eq1} \\
\Var[X] &= \E[X^2] - (\E[X])^2 \label{eq2} 
\end{align}
Now, for the variance of $(X_1 - 2X_2)$
\begin{align}
\Var[X_1 - 2X_2] &= \E[(X_1 - 2X_2)^2] - (\E[X_1 - 2X_2])^2 \nonumber \\
&= \E[X_1^2 + 4X_2^2 - 4X_1X_2] \nonumber \\
&- (\E[X_1] - 2\E[X_2])^2 \nonumber \\
&= \E[X_1^2] - (\E[X_1])^2 + 4(\E[X_2^2]-(\E[X_2])^2) \nonumber \\
&- 4\E[X_1X_2]  + 4\E[X_1]\E[X_2]
\end{align}
Since the variables are independent:
\begin{align}
\E[X_1X_2] = \E[X_1]\E[X_2]
\end{align}
Substituting equations \eqref{eq1} and \eqref{eq2}, we get:
\begin{align}
\Var[X_1 - 2X_2] &= \Var[X_1] + 4(\Var[X_2]) \nonumber \\
&- 4\E[X_1][X_2] + 4\E[X_1][X_2] \nonumber \\
&= \lambda_1 + 4\lambda_2 = 2 + 4(3) = 14
\end{align}
Hence option (a) 14 is correct.
\end{document}
